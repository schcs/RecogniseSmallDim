\documentclass[12pt]{amsart}
\usepackage{amssymb}
\usepackage{latexsym}
\usepackage{amsfonts}
\usepackage{amsmath}
\usepackage{amssymb}
\usepackage{showkeys}
\usepackage{lscape}
\usepackage{pdflscape}
%\usepackage{times}
%\usepackage{hyperref}
\usepackage{algorithm2e}
\usepackage{afterpage}
\usepackage{color}


\oddsidemargin 0pt
\evensidemargin 0pt
\textheight 8.1in \textwidth 6.3in
\setlength{\parskip}{5pt}
\renewcommand{\baselinestretch}{1.2}



\relpenalty=10000
\binoppenalty=10000
\tolerance=500





\newtheorem{theorem}{Theorem}[section]
\newtheorem{proposition}[theorem]{Proposition}
\newtheorem{lemma}[theorem]{Lemma}
\newtheorem{corollary}[theorem]{Corollary}

\theoremstyle{definition}


\newtheorem{remark}[theorem]{Remark}
\newtheorem{definition}[theorem]{Definition}
\newtheorem{example}[theorem]{Example}




\mathsurround=1pt
\headheight 14pt
\renewcommand{\baselinestretch}{}
%\parskip 3pt

\newcommand{\V}{\mathcal V}
\newcommand{\W}{\mathcal W}
\renewcommand{\S}{\mathcal S}
\newcommand{\B}{\mathcal B}
\newcommand{\U}{\mathcal U}
\newcommand{\T}[2]{X\!\left(#1,#2\right)}
\newcommand{\lre}[3]{T_{#1,#2}(#3)}
\newcommand{\lree}[2]{T_{#1,#2}}


\renewcommand{\b}[1]{b^{(#1)}}
\renewcommand{\v}[1]{v^{(#1)}}
\newcommand{\trgr}[1]{\left<\!\left<#1\right>\!\right>}
\newcommand{\qq}[1]{{#1}}
\newcommand{\tr}[1]{#1^{\rm t}}

\newcommand{\biform}[2]{(#1,#2)}

\renewcommand{\leq}{\leqslant}
\renewcommand{\geq}{\geqslant}

\newcommand{\gl}[2]{{\sf GL}(#1,#2)}
\newcommand{\SL}[2]{{\sf SL}(#1,#2)}
\newcommand{\Sp}[2]{{\sf Sp}(#1,#2)}
\newcommand{\psl}[2]{{\sf PSL}(#1,#2)}
\newcommand{\SU}[2]{{\sf SU}(#1,#2)}
\newcommand{\SX}[2]{{\sf SX}(#1,#2)}
\newcommand{\GX}[2]{{\sf GX}(#1,#2)}
\newcommand{\psp}[2]{{\sf PSp}(#1,#2)}
\newcommand{\psu}[2]{{\sf PSU}(#1,#2)}
\newcommand{\pomega}[2]{{\sf \Omega}(#1,#2)}
\newcommand{\pomegap}[2]{{\sf P\Omega^+}(#1,#2)}
\newcommand{\pomegam}[2]{{\sf P\Omega^-}(#1,#2)}
\newcommand{\X}{\mathcal X}
\newcommand{\F}{\mathbb F}
\newcommand{\conj}{\,\widehat{\ }\,}

\newcommand{\cent}[1]{{\sf Z}(#1)}
\newcommand{\class}[2]{{\sf Cl}(#1,#2)}
\newcommand{\classv}[1]{{\sf Cl}(#1)}

\begin{document}



\title{Recognising small-dimensional representations of classical groups}
\author{CS}
\address
{Departamento de Matem\'atica\\
Instituto de Ci\^encias Exatas\\
Universidade Federal de Minas Gerais
Belo Horizonte, MG, Brasil}
\email{csaba@ufmg.br}


\maketitle

\begin{abstract}
\end{abstract}



\section{introduction}
We describe some practical methods to recognise
small-dimensional representations of classical groups over finite-fields.

\section{Alternating square}

\subsection{The structure of $\boldsymbol{V\wedge V}$}
Suppose that $G$ is a classical group over a finite field $\F_q$
with characteristic $p$ acting on its natural module
$V=\{v_1,\ldots,v_d\}$. Set
$W:=V\wedge V$. The natural ordered basis of $W$ will be taken to be
\[
v_1\wedge v_2,\ldots,v_1\wedge v_d,v_2\wedge v_3,\ldots,v_{d-1}\wedge v_d.
\]
We denote by $B_V$ the natural bilinear form preserved by $G$ on $V$.
In the cases different from SL, we assume that $(v_1,v_d),\ldots,(v_d,v_{d+1})$
are hyperbolic pairs.



\begin{lemma}\label{lem:str_notsp}
  The following are valid.
  \begin{enumerate}
  \item The group $G$ preserves the bilinear form $B_W$ defined by
    \[
    B_W(v_i\wedge v_j,v_r\wedge v_s)=B_V(v_1,v_r)B_V(v_j,v_s).
    \]
  \item If $G$ is SL, then $B_W(\cdot,\cdot)=0$; otherwise $B_W$ is non-degenerate.
  \item If $G$ is not Sp, then $G$ is irreducible on $W$.
  \end{enumerate}
\end{lemma}

In the case of Sp, $G$ is not irreducible on $W$. Set
\[
W_0=\left<v_1\wedge v_{d}+v_2\wedge v_{d-1}+\cdots+v_{d}\wedge v_{d+1}\right>
\]
and let $W_1$ be the subspace generated by the set
    \begin{align*}
    &\{v_i\wedge v_j\mid i<j\mbox{ and }i+j\neq d+1\}\cup\\
    &\{v_1\wedge v_d-v_{d/2}\wedge v_{d/2+1},v_2\wedge v_{d-1}
    -v_{d/2}\wedge v_{d/2+1},\ldots,v_{d/2-1}\wedge v_{d/2+2}-v_{d/2}\wedge v_{d/2+1}\}
    \end{align*}

\begin{lemma}
  If $G$ is Sp, then the subspaces $W_0$ and $W_1$ are $G$-submodules. Furthermore the following
  hold.
  \begin{enumerate}
    
  \item If $p \nmid d$, then $W_1$ is $G$-irreducible and the $G$-module
    $W$ can be written as the direct
    sum  $W=W_0\oplus W_1$
  \item If $p\mid d$, then $W_0\leq W_1$ and
    $W_1/W_0$ is an irreducible $G$-module.
  \end{enumerate}
\end{lemma}

\subsection{Recognising $\boldsymbol V\wedge V$ for irreducible
  $\boldsymbol G$}
Suppose in this section that $G$ is not Sp. Then by Lemma~\ref{lem:str_notsp}, $G$
is irreducible on $W$.

\begin{lemma}
  Let $x\in G$ be an involution with $-1$-eigenspace of dimension $k$
  where $2\leq k\leq d-1$. Let $C$ be the centralizer $C_G(x)$.
  $W=U_1\oplus U_2\oplus U_3$ where $U_1$ and $U_2$ are isomorphic to
  the exterior squares of $V_k$ and $V_{d-k}$ and $U_3\cong V_k\otimes V_{d-k}$. 
\end{lemma}

Suppose that we have determined bases of $V_k\wedge V_k$,
$V_{d-k}\wedge V_{d-k}$, and $V_k\times V_{d-k}$ in the form
\begin{align*}
  &\{ v_i\wedge v_j\mid i<j\mbox{ and }i,j\in\{k+1,\ldots,d\}\}\\
  &\{ v_i\wedge v_j\mid i<j\mbox{ and }i,j\in\{k+1,\ldots,d\}\}\\
  &\{ \delta(v_i \otimes v_k)\mid i\in\{1,\ldots,k\},\ j\in\{k+1,\ldots,d\}\}.
  \end{align*}
To find the basis of $V$, we need to determine the scalar $\delta$.

Let $X$ be the matrix of $g\in G$ given in its action on $W$ with respect to
the basis determined above. Suppose that $A$ and $B$ are the $3\times 3$
matrices  that correspond to the entries $\{e_{1,1},e_{1,k+1},e_{2,k+1}\}$
and $\{e_{1,k+1},e_{2,k+1},e_{k+1,k+2}\}$, respectively. Then the $\{1,2,k+1\}$ and
the $\{1,k+1,k+2\}$ minors of the matrix $Y$ corresponding to the same
element, but considered in its action on $V$, can be recovered as $\bigwedge A$
and $\bigwedge B$. By multiplying $\bigwedge B$ with the scalar
$\bigwedge A[e_1,e_1](\bigwedge B[e_1,e_1])^{-1}$, we may assume that
$\bigwedge A[e_1,e_1]=\bigwedge B[e_1,e_1]$ (this step corrects if there
is some problem with constant on the
bases of $\bigwedge V_k$ or $\bigwedge V_{d-k}$. Now
\[
\delta^2=\bigwedge B[1,2]/\bigwedge A[1,3].
\]


\end{document}


